\documentclass{SIGCHI2015LaTex/sigchi}

% Use this command to override the default ACM copyright statement (e.g. for preprints). 
% Consult the conference website for the camera-ready copyright statement.


%% EXAMPLE BEGIN -- HOW TO OVERRIDE THE DEFAULT COPYRIGHT STRIP -- (July 22, 2013 - Paul Baumann)
% \toappear{Permission to make digital or hard copies of all or part of this work for personal or classroom use is granted without fee provided that copies are not made or distributed for profit or commercial advantage and that copies bear this notice and the full citation on the first page. Copyrights for components of this work owned by others than ACM must be honored. Abstracting with credit is permitted. To copy otherwise, or republish, to post on servers or to redistribute to lists, requires prior specific permission and/or a fee. Request permissions from permissions@acm.org. \\
% {\emph{CHI'14}}, April 26--May 1, 2014, Toronto, Canada. \\
% Copyright \copyright~2014 ACM ISBN/14/04...\$15.00. \\
% DOI string from ACM form confirmation}
%% EXAMPLE END -- HOW TO OVERRIDE THE DEFAULT COPYRIGHT STRIP -- (July 22, 2013 - Paul Baumann)


% Arabic page numbers for submission. 
% Remove this line to eliminate page numbers for the camera ready copy
% \pagenumbering{arabic}


% Load basic packages
\usepackage{balance}  % to better equalize the last page
\usepackage{graphics} % for EPS, load graphicx instead
%\usepackage{times}    % comment if you want LaTeX's default font
\usepackage{url}      % llt: nicely formatted URLs

% llt: Define a global style for URLs, rather that the default one
\makeatletter
\def\url@leostyle{%
  \@ifundefined{selectfont}{\def\UrlFont{\sf}}{\def\UrlFont{\small\bf\ttfamily}}}
\makeatother
\urlstyle{leo}


% To make various LaTeX processors do the right thing with page size.
\def\pprw{8.5in}
\def\pprh{11in}
\special{papersize=\pprw,\pprh}
\setlength{\paperwidth}{\pprw}
\setlength{\paperheight}{\pprh}
\setlength{\pdfpagewidth}{\pprw}
\setlength{\pdfpageheight}{\pprh}

% Make sure hyperref comes last of your loaded packages, 
% to give it a fighting chance of not being over-written, 
% since its job is to redefine many LaTeX commands.
\usepackage[pdftex]{hyperref}
\hypersetup{
pdftitle={SIGCHI Conference Proceedings Format},
pdfauthor={LaTeX},
pdfkeywords={SIGCHI, proceedings, archival format},
bookmarksnumbered,
pdfstartview={FitH},
colorlinks,
citecolor=black,
filecolor=black,
linkcolor=black,
urlcolor=black,
breaklinks=true,
}

% create a shortcut to typeset table headings
\newcommand\tabhead[1]{\small\textbf{#1}}


% End of preamble. Here it comes the document.
\begin{document}

\title{Transmedia games and activity-tracking wearables that encourage fitness in children}

\numberofauthors{3}

\author{
\alignauthor {
Adam Tindale$^1$, Michael Cumming$^2$, Sara Diamond$^3$}
\affaddr{$^1$ $^2$ $^3$ OCAD University Toronto, ON M5T 1W1 Canada}
\email{atindale@faculty.ocadu.ca}, 
\email{mcumming@ocadu.ca},
\email{sdiamond@ocadu.ca}
}

%Correspondence should be addressed to: 
%\href
%{mailto:atindale@faculty.ocadu.ca}{atindale@faculty.ocadu.ca},
%\href
%{mailto:mcumming@ocadu.ca}{mcumming@ocadu.ca},
%\href
%{mailto:sdiamond@ocadu.ca}{sdiamond@ocadu.ca}
%}


%}
%  \alignauthor Adam Tindale\\
%    \affaddr{OCAD University}\\
%    \affaddr{Toronto, ON M5T1W1 Canada}\\
%    \email{atindale@faculty.ocadu.ca}\\
%  \alignauthor Michael Cumming\\
%    \affaddr{OCAD University}\\
%    \affaddr{Toronto, ON M5T1W1 Canada}\\
%    \email{mcumming@ocadu.ca}\\   
%  \alignauthor Sara Diamond\\
%    \affaddr{OCAD University}\\
%    \affaddr{Toronto, ON M5T1W1 Canada}\\
%    \email{sdiamond@ocadu.ca}\\
%}

\maketitle

\begin{abstract}

Studies show that the degree to which children are physically active is a good predictor of overall fitness. Simple wearable devices can measure this quantity fairly accurately. However, children need more incentive than simple activity tracking to encourage them to be active and healthy. We propose that participation in game narratives provides such incentive. We have built a wearable device with which children can play an engaging transmedia collection game called Time Tremors. Physical encouragement is integrated into the game narrative so that children participate unselfconsciously and engage kinaesthetically with narratives, artifacts and cultural venues.

\end{abstract}

\keywords{
transmedia, wearable technology, exergaming, fitness tracking, movement biometrics
}

\category{H.5.m.}{Information Interfaces and Presentation (e.g. HCI)}{Miscellaneous}

\section{Introduction}
This work concerns the intersection of transmedia game design and wearables that encourage activity in children. Game-playing is a popular and time-consuming pastime for many children. Currently, obesity is a threat to the health of children \cite{ebbeling2002childhood}. Most child-oriented games, even those played on wearables, encourage sedentary play \cite{graves2008energy}. Some would argue that widespread yet sedentary game-play is a threat to the fitness of children. Games and game-oriented wearables should not decrease the health or fitness to whom they are marketed. Gaming for children should at least reach the threshold of 'do no harm'. 

Wearables with their many sensors and rich potential as gaming platforms can be designed in to encourage movement in their wearers. The issue is how to encourage movement in a way that will work both in the short and long term. Activity, if it is to have an effect on fitness, must entail some degree of long-term behavioural change. This is a tall order for wearables whose likely lifespan may be short. 

Activity tracking wearables usually assume implicitly that representation of activity patterns will provide sufficient motivation to improve overall fitness. The idea is that if you see graphically how your behaviour is changing then you will be motivated to continue your process of improvement. This may be true for some but as a recipe for fitness improvement for a population who have various motivations it appears naive. 

Greater activity has to be connected to some additional goal beyond representation that the player finds meaningful. Our approach is to integrate player movement with the exploration of transmedia narratives. We believe that narratives in their many forms have the depth and power to motivate behaviour. The behavioural modification implied by building healthy games and wearables for children need not be explicit. If the game narrative is designed in such a way that player movement is an organic aspect of the story then connecting player movement to game play does not become intrusive.

\section{The transmedia gaming environment of Time Tremors}
Many children enjoy playing screen-based games. The reason that spend so much time playing them is that they find them cognitively engaging and fun \cite{gee2003video}. Our device works in conjunction with a transmedia and location-based game called Time Tremors. Being a transmedia game, Time Tremors can be viewed on several platforms including TV, and online on computers, smart phones and tablets. Transmedia is that which spans multiple devices and media channels \cite{phillips2012creators}. These narratives are authored by writers intentionally, with explicit and carefully engineered connection points between media platforms. Despite these transition points, the integrity of the narrative must remain intact. To continue viewing or participating in the narrative, the viewer has the opportunity to switch devices without losing the narrative thread. The transmedia narrative connection adds engagement for the player to an interaction with a wearable. Increased physical activity is an intended side effect of playing the game. Compared to how sedentary behaviour is an unintended some effect of playing most other games.

\section{The promise of 'exergames'}
Exergames on ones that encourage or require movement in their players. In order to work they both have to be shown to actually require physical activity, and they also have to be adequately engaging and interesting to provide incentive for people to play them \cite{whitehead2010exergame}. Activity that is expected to be have a significant fitness effect tends to be rather strenuous and take place over a long period. The results of a new exercise regime may take months to appear. An effective exergame must encourage people to start moving more and it must be sufficiently interesting for people to continue working at it. Like gym memberships, the novelty of exercising with or without a device often wears off quickly. This is a difficult standard for a game or device to achieve: work as real exercise and sufficiently motivate people to use and benefit from them over an extended period. The most prominent examples of exergames are those played on the Nintendo Wii. 

Whitehall, et al. survey the number of calories expended while playing exergames and found that games exhibit a wide range of values. Some are quite effective while others do very little \cite{whitehead2010exergame}. Typically, these games are marketed towards children who find them fun to play. There is also a large market for exergames tailored towards the elderly, where the negative effects of sedentary lifestyles are even more problematic than with children \cite{vonstad2014exergaming}.

There is a strong correlation between heart rate and energy expenditure \cite{whitehead2010exergame}. Activities that engage the whole body tend to be more effective for fitness than ones that engage individual limbs \cite{mortazavi2014near}. One way of encouraging whole body movement is to attach several sensors to player's limbs. Some exergames (e.g. Wii) allow pseudo-movement: minor movement read incorrectly by devices' sensors as major body movement. Exergames should be designed to prevent this kind of `cheating'. Measuring and cross-referencing aspects of activity, such as body and limb movement, and heart rate more accurately measures real activity that might have a real effect on fitness.  

\section{Biometrics on wearables}
Wrist wearables are typically small devices on which normal touch interfaces are typically problematic due to their small size. Using the device as part of a hand gesture recognition system overcomes the small size limitation \cite{kim2007gesture}. Our device is used a simple platform for displaying the degree of physically activity of the player and a housing for the sensors (the accelerometer and heart rate sensor) that measure this activity. Measuring biometrics is an area of increasing interest, especially in the medical fields. Biosensors measure physiological parameters like blood pressure, heart rate, oxygen saturation of the blood, respiration rate, electrical muscle activity in limbs and the heart, body and skin temperature, etc. This technology is expected to have a major effect on real-time monitoring of health and fitness for personal information and medical diagnostic purposes  \cite{pantelopoulos2010survey}. This technology is particularly suitable for monitoring patients outside of hospitals at their homes as they go about their normal routines. In non-medical situations biometric sensors are also useful in games in which activity, fitness and movement play a role.  

\section{Time Tremors collection game}
Our device is an device that enables the child to interact and play with the transmedia, location and collection-based game called Time Tremors \cite{tindale2014wearable}. Children collect virtual artifacts that they discover while on location-based treasure hunts in museums and other venues. In the Time Tremors game children must maintain a level of a game-specific quantity called 'time energy'. If the player has sufficient time energy players are able to collect 'time treasures'. Time treasures are the item of interest to be collected within the game. If sufficient treasures are collected, in certain combinations, then other opportunities to collect additional treasures are presented to the player. There is also a transmedia connection in that players can collect treasures while watching TV episodes connected to the game, navigating successfully through a flying game, or solving puzzles. The game also include supernatural elements, meant to appeal to the imagination of its intended audience, such as ghosts from the past, robotic insects and talking bears.

These treasures are real artifacts form history, such as the First Telephone, the First Printing Press and ones from the world of fiction, past, present and future, such a Jedi Helmet from the Star Wars franchise or an imaginary Po-Pod craft from 2320 AD. The purpose of these artifacts and their combinations is to  teach a certain level of cultural awareness in children and to weave a stories in which the game's prime antagonist Ms. Bugly thwarts children's attempts to gather valuable artifacts. The term Time Tremors refers to a worm-hole like situation in which children are transported through time to witness past and future epochs. While transported children also have the opportunity to gather Time Treasures. Children learn about history by collecting artifacts from history. They also imagine future worlds by collecting artifacts from these worlds. 

\section{Wearable that interacts with Time Tremors narratives}
Our device is a wrist wearable that incorporates a wireless bluetooth microprocessor, an electrocardiogram (ECG) type heart sensor and an accelerometer that measures movement and accelerative forces. Accelerometers that measure acceleration in three exes are common in wearables to measure activity of various types \cite{alshurafa2014designing}. The purpose for using these two sensors working together is to get a composite measure that indicates the overall physical activity of the child wearing the device. The player needs to have a heart rate above a certain threshold and also have physical movement measured by the device's accelerometer above a certain level. 

The Tim Tremors game itself has a certain high level of energy to appeal to young players. High physical activity while wearing the wrist device required to progress in the game corresponds to the overall energy level of the narrative and its fanciful constructions. Through this connection of high energy actual movement with high energy narratives the incentive for children to move is less forced and therefore more likely to be effective. 

This approach towards 'fitness as a side effect of game-playing' does however demand that narratives are invented which both exhibit high energy and also encourage high energy movement in their viewers. Narratives, except those of epics, tend not to last long and tend to be expensive to produce. Games, even the most compelling, are of finite length. Healthy lifestyles in children shouldn't be dependent on a continuous stream of narratives that encourage them to be active. However, getting children to lead more active lives should be a major concern wherever children lead sedentary lives. 

\section{Conclusion}
Children spend an inordinately large number of hours playing games. One approach to increasing the value of this time spent is to convert some of them into so-called 'exergames'. This type of game has the promise of increasing activity and fitness in the children who play them. It is not difficult to engage children in a game for short period of time, however, healthy activity habits require a long-term commitment and high levels of physical and intellectual engagement. Our approach is to integrate and to increase the engagement of players using the device of transmedia gaming. Improved fitness through activity is not achieved by the goal of fitness, which for children may be a minor goal, but by the desire of players to explore engaging game-driven narratives in which player movement is integrated. 

\section{Acknowledgments}

We would like to thank NSERC, ISTP, OCADU and others for their support.

% Balancing columns in a ref list is a bit of a pain because you
% either use a hack like flushend or balance, or manually insert
% a column break.  http://www.tex.ac.uk/cgi-bin/texfaq2html?label=balance
% multicols doesn't work because we're already in two-column mode,
% and flushend isn't awesome, so I choose balance.  See this
% for more info: http://cs.brown.edu/system/software/latex/doc/balance.pdf
%
% Note that in a perfect world balance wants to be in the first
% column of the last page.
%
% If balance doesn't work for you, you can remove that and
% hard-code a column break into the bbl file right before you
% submit:
%
% http://stackoverflow.com/questions/2149854/how-to-manually-equalize-columns-
% in-an-ieee-paper-if-using-bibtex
%
% Or, just remove \balance and give up on balancing the last page.
%
\balance

%\section{References format}
%References must be the same font size as other body text.
% REFERENCES FORMAT
% References must be the same font size as other body text.

\bibliographystyle{SIGCHI2015LaTex/acm-sigchi}
\bibliography{../refs}

\end{document}
