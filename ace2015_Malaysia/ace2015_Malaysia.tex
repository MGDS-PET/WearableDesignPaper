\documentclass{SIGCHI2015LaTex/sigchi}

% Use this command to override the default ACM copyright statement (e.g. for preprints). 
% Consult the conference website for the camera-ready copyright statement.


%% EXAMPLE BEGIN -- HOW TO OVERRIDE THE DEFAULT COPYRIGHT STRIP -- (July 22, 2013 - Paul Baumann)
% \toappear{Permission to make digital or hard copies of all or part of this work for personal or classroom use is granted without fee provided that copies are not made or distributed for profit or commercial advantage and that copies bear this notice and the full citation on the first page. Copyrights for components of this work owned by others than ACM must be honored. Abstracting with credit is permitted. To copy otherwise, or republish, to post on servers or to redistribute to lists, requires prior specific permission and/or a fee. Request permissions from permissions@acm.org. \\
% {\emph{CHI'14}}, April 26--May 1, 2014, Toronto, Canada. \\
% Copyright \copyright~2014 ACM ISBN/14/04...\$15.00. \\
% DOI string from ACM form confirmation}
%% EXAMPLE END -- HOW TO OVERRIDE THE DEFAULT COPYRIGHT STRIP -- (July 22, 2013 - Paul Baumann)


% Arabic page numbers for submission. 
% Remove this line to eliminate page numbers for the camera ready copy
% \pagenumbering{arabic}


% Load basic packages
\usepackage{balance}  % to better equalize the last page
\usepackage{graphics} % for EPS, load graphicx instead
%\usepackage{times}    % comment if you want LaTeX's default font
\usepackage{url}      % llt: nicely formatted URLs

% llt: Define a global style for URLs, rather that the default one
\makeatletter
\def\url@leostyle{%
  \@ifundefined{selectfont}{\def\UrlFont{\sf}}{\def\UrlFont{\small\bf\ttfamily}}}
\makeatother
\urlstyle{leo}


% To make various LaTeX processors do the right thing with page size.
\def\pprw{8.5in}
\def\pprh{11in}
\special{papersize=\pprw,\pprh}
\setlength{\paperwidth}{\pprw}
\setlength{\paperheight}{\pprh}
\setlength{\pdfpagewidth}{\pprw}
\setlength{\pdfpageheight}{\pprh}

% Make sure hyperref comes last of your loaded packages, 
% to give it a fighting chance of not being over-written, 
% since its job is to redefine many LaTeX commands.
\usepackage[pdftex]{hyperref}
\hypersetup{
pdftitle={SIGCHI Conference Proceedings Format},
pdfauthor={LaTeX},
pdfkeywords={SIGCHI, proceedings, archival format},
bookmarksnumbered,
pdfstartview={FitH},
colorlinks,
citecolor=black,
filecolor=black,
linkcolor=black,
urlcolor=black,
breaklinks=true,
}

% create a shortcut to typeset table headings
\newcommand\tabhead[1]{\small\textbf{#1}}


% End of preamble. Here it comes the document.
\begin{document}

\title{Using transmedia games and wearables to encourage fitness in children}

\numberofauthors{3}

\author{
\alignauthor {
Adam Tindale$^1$, Michael Cumming$^2$, Sara Diamond$^3$}
\affaddr{$^1$ $^2$ $^3$ OCAD University Toronto, ON M5T 1W1 Canada}
\email{atindale@faculty.ocadu.ca}, 
\email{mcumming@ocadu.ca},
\email{sdiamond@ocadu.ca}
}

%Correspondence should be addressed to: 
%\href
%{mailto:atindale@faculty.ocadu.ca}{atindale@faculty.ocadu.ca},
%\href
%{mailto:mcumming@ocadu.ca}{mcumming@ocadu.ca},
%\href
%{mailto:sdiamond@ocadu.ca}{sdiamond@ocadu.ca}
%}


%}
%  \alignauthor Adam Tindale\\
%    \affaddr{OCAD University}\\
%    \affaddr{Toronto, ON M5T1W1 Canada}\\
%    \email{atindale@faculty.ocadu.ca}\\
%  \alignauthor Michael Cumming\\
%    \affaddr{OCAD University}\\
%    \affaddr{Toronto, ON M5T1W1 Canada}\\
%    \email{mcumming@ocadu.ca}\\   
%  \alignauthor Sara Diamond\\
%    \affaddr{OCAD University}\\
%    \affaddr{Toronto, ON M5T1W1 Canada}\\
%    \email{sdiamond@ocadu.ca}\\
%}

\maketitle

\begin{abstract}

Studies show that the degree to which children are physically active is a good predictor of overall fitness. Simple wearable devices can measure this quantity fairly accurately. However, children need more incentive than simple activity tracking to encourage them to be active and healthy. We propose that participation in game narratives provides such incentive. We have built a wearable device that encourages healthy movement while playing an engaging transmedia and collection-type game. The more physical encouragement is integrated into the game narrative the greater is the chance that children will participate unselfconsciously and engage kinaesthetically with narratives, artifacts and cultural venues.

\end{abstract}

\keywords{
Transmedia, wearable technology, fitness tracking, biometrics
}

\category{H.5.m.}{Information Interfaces and Presentation (e.g. HCI)}{Miscellaneous}

\section{Introduction}
This work concerns the intersection of transmedia game design and wearables that encourage activity in children. Most games oriented towards children, even those played on wearables, encourage sedentary playing positions. This is a problem with many games for children: they tend to encourage a sedentary lifestyle. Since game-playing is such a popular activity for many children, game-playing when sedentary becomes a threat to the fitness of children. Games and game-oriented wearables cannot be seen to decrease the health or fitness to whom they are marketed. Gaming for children should at least reach the threshold of 'do no harm.'

Wearables, with their many sensors and potential as gaming platforms, can also be designed in to encourage movement. If children are less sedentary their overall fitness is expected to improve. Since, obesity in children is currently a major concern, this approach could be helpful towards health and fitness in children. The goal of this research is produce intellectually and physically engaging wearable devices and games involving narratives for children designed to encourage activity and fitness. 

Many children enjoy playing games. The reason that spend so much time playing them is that they find them engaging and fun. Our device works in conjunction with a transmedia and location-based game called Time Tremors. Being a transmedia game, Time Tremors can be viewed on several platforms including TV, and online on computers, smart phones and tablets. Transmedia is that which spans multiple devices and media channels. These narratives are authored by writers intentionally, with explicit and carefully engineered connection points between media platforms. Despite these transition points, the integrity of the narrative must remain intact. To continue viewing or participating in the narrative, the viewer has the opportunity to switch devices without losing the narrative thread. The transmedia narrative connection adds engagement for the player to an interaction with a wearable. Increased physical activity is an intended side effect of playing the game. Compare with how sedentary behaviour is an unintended some effect of playing most other games.

\section{The promise of 'exergames'}
Exergames on ones that encourage or require movement in their players. In order to work they both have to be shown to actually require physical activity, and they also have to adequately engaging and interesting to provide incentive for people to play them \cite{whitehead2010exergame}. Activity that is expected to be have a significant fitness effect tends to be rather strenuous. The results of a new exercise regime may takes months to appear. 

An effective exergame must encourage people to start moving more and it must be sufficiently interesting for people to continue working at it. Like gym memberships, the novelty of exercising with or without a device often wears off quickly. This is a rather difficult standard for a game or device to achieve: work as real exercise and sufficiently motivate people to use and benefit from them. The most prominent examples of exergames are those played on the Nintendo Wii. 

Whitehall, et al. survey the number of calories expended while playing exergames and found that there is a wide range of values. Typically, these games are marketed towards children, who find them fun to play. There is also a large market for exergames -- if they actually do encourage fitness over the long term -- for the elderly, where the negative effects of sedentary lifestyles are even more problematic than with children \cite{vonstad2014exergaming}.

There is a strong correlation between heart rate and energy expenditure\cite{whitehead2010exergame}. Activities that engage the whole body tend to be more effective for fitness than ones that engage individual limbs \cite{mortazavi2014near}. One way of encouraging while body movement is to attach several sensors to player's limbs. Some exergames (e.g. Wii) allow pseudo-movement: minor movement read incorrectly by device sensors as major body movement. Exergames should be designed to prevent 'cheating.' Measuring more aspects of activity, such as body and limb movement, and heart rate more accurately measures real activity that will likely to have a real effect on fitness.  

\section{Time Tremors collection game}
Our device is an device that enables the child to interact and play with the transmedia, location and collection-based game called Time Tremors. Time Tremors is a collection-type game. Children collect virtual artifacts that they discover while on location-based treasure hunts in museums, and other venues. Children must maintain a level of a game-specific quantity called 'time energy.' Time energy is gathered as one maintains a level of physical activity while wearing the device. If the player has sufficient time energy players are able to collect 'time treasures.' Time treasures are the item collected within the game. If sufficient treasures are collected, in certain combinations, then other opportunities to collect additional treasures are presented to the player. There is also a transmedia connection in that players can collect treasures while watching TV episodes connected to the game, navigating successfully through a flying game, or solving puzzles.

\section{Our wearable wrist device}
Our device is a wrist wearable that incorporates a wireless bluetooth microprocessor, an electrocardiogram (ECG) type heart sensor and an accelerometer that measures movement and accelerative forces. Accelerometers that measure acceleration in three exes are common in wearables to measure activity of various types \cite{alshurafa2014designing}.The purpose for using these two sensors working together is to get a composite measure that indicates the overall physical activity of the child wearing the device. The player needs to have a heart rate above a certain threshold and also have physical movement measured by the device's accelerometer above a certain level. 

\section{Biometrics on wearables}
Measuring biometrics is an area of increasing interest, especially in the medical fields. Biosensors measure physiological parameters like blood pressure, heart rate, oxygen saturation of the blood, respiration rate, electrical muscle activity in limbs and the heart, body and skin temperature, etc. This technology is expected to have a major effect on real-time monitoring of health and fitness for personal information and medical diagnostic purposes  \cite{pantelopoulos2010survey}. This technology is particularly suitable for monitoring patients outside of hospitals at their homes as they go about their normal routines. In non-medical situations biometric sensors are also useful in games in which activity, fitness and movement play a role.  

\section{Games on wrist wearables}
Wrist wearables are typically small devices on which normal touch interfaces are typically problematic due to their small size. Using the device as part of a hand gesture recognition system overcomes the small size limitation \cite{kim2007gesture}. Our device is used a simple platform for displaying the degree of physically activity of the player and a housing for the sensors (the accelerometer and heart rate sensor) that measure this activity. 



%Basic idea of this combination:
%- If children play the 
%- It integrates an existing transmedia property called 'Time Tremors'
%- Goal: engage children with a transmedia, encourage them to engage in movement
%
%Wearable devices are useful for recording biometric data. The ebbs, flows, mental and physical capacities of people's bodies inform their lives. When viewed over time, and organized into a useful format, this data can be seen as a type of personal narrative. It is not just a stream of unrelated data; it can be structured to have a 'narrative' arc. 
%
%If bodies are capable of becoming sources of important narratives then to measure aspects of people's bodies, their mobility, their actions, their positioning in space and time, in a structured way, such narratives could possibly be useful for explaining people's lives to themselves.
%
%
%
%This research examines the connection between biometric data and how it might be structured into narratives similar to interactive transmedia narratives. 
%Both narrative and transmedia are important, though sometimes quite hidden, aspects of modern lives. Narratives, stories that people construct to explain their actions, inform all aspects of interior and social life. Internal narratives seem important for our psychological makeup. 
%
%Certain points in any narrative that are more important than others. For instance, in a media property called Time Tremors (a series involving pre-teen children and their cosmically evil teacher) it is when ominous, supernatural things start happening to the protagonists, such as when toy bears start talking, bugs invade a classroom or when a so-called time tremor is triggered. These are special inflection points within the narrative that demand special attention. 
%
%In a similar fashion there are also inflection points in internal narratives; these are points in people's lives when life takes a different direction. Wearable device are useful by their ability to measure and identify these important 'inflection points.'
%
%In transmedia the viewer has some ability to exercise choice and to customize their narrative experience. Transmedia narratives must be designed to allow for this kind of non-prescribed navigation. This gives an inherent modularity to its content.
%
%Narratives generated by bodies as they navigate the world is similar to narratives written to span multiple media venues: both are modular, customizable, and depend on the personal experiences of the viewers. 

%\section{Internal Narratives}
%
%Internal narratives are ones were construct within ourselves. People create narratives by reflecting on the processes within their lives. This process requires that people engage conceptually and cognitively with their experience. People order and make sense of their lives in terms of narrative.

\section{Conclusion}
Children spend an inordinately large number of hours playing games. One approach to increasing the value of this time spent is to convert some of them into so-called 'exergames.' This type of game has the promise of increasing activity and fitness in the children who play them. It is not difficult to engage children in a game for short period of time. However, healthy habits with respect to activity requires a long-term commitment and high levels of physically and intellectual engagement. Our approach is to integrate and to increase the engagement of players using the device of transmedia gaming. The narrative appeal of transmedia should increase the incentive of physical activity, not by the appeal of heightened fitness, but by the desire of players to explore narratives. 

\section{Acknowledgments}

We would like to thank NSERC, ISTP, OCADU and others for their support.

% Balancing columns in a ref list is a bit of a pain because you
% either use a hack like flushend or balance, or manually insert
% a column break.  http://www.tex.ac.uk/cgi-bin/texfaq2html?label=balance
% multicols doesn't work because we're already in two-column mode,
% and flushend isn't awesome, so I choose balance.  See this
% for more info: http://cs.brown.edu/system/software/latex/doc/balance.pdf
%
% Note that in a perfect world balance wants to be in the first
% column of the last page.
%
% If balance doesn't work for you, you can remove that and
% hard-code a column break into the bbl file right before you
% submit:
%
% http://stackoverflow.com/questions/2149854/how-to-manually-equalize-columns-
% in-an-ieee-paper-if-using-bibtex
%
% Or, just remove \balance and give up on balancing the last page.
%
\balance

\section{References format}
References must be the same font size as other body text.
% REFERENCES FORMAT
% References must be the same font size as other body text.

\bibliographystyle{SIGCHI2015LaTex/acm-sigchi}
\bibliography{../refs}
\end{document}
