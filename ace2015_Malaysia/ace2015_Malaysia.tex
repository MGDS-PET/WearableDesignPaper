\documentclass{SIGCHI2015LaTex/sigchi}

% Use this command to override the default ACM copyright statement (e.g. for preprints). 
% Consult the conference website for the camera-ready copyright statement.


%% EXAMPLE BEGIN -- HOW TO OVERRIDE THE DEFAULT COPYRIGHT STRIP -- (July 22, 2013 - Paul Baumann)
% \toappear{Permission to make digital or hard copies of all or part of this work for personal or classroom use is 	granted without fee provided that copies are not made or distributed for profit or commercial advantage and that copies bear this notice and the full citation on the first page. Copyrights for components of this work owned by others than ACM must be honored. Abstracting with credit is permitted. To copy otherwise, or republish, to post on servers or to redistribute to lists, requires prior specific permission and/or a fee. Request permissions from permissions@acm.org. \\
% {\emph{CHI'14}}, April 26--May 1, 2014, Toronto, Canada. \\
% Copyright \copyright~2014 ACM ISBN/14/04...\$15.00. \\
% DOI string from ACM form confirmation}
%% EXAMPLE END -- HOW TO OVERRIDE THE DEFAULT COPYRIGHT STRIP -- (July 22, 2013 - Paul Baumann)


% Arabic page numbers for submission. 
% Remove this line to eliminate page numbers for the camera ready copy
% \pagenumbering{arabic}


% Load basic packages
\usepackage{balance}  % to better equalize the last page
\usepackage{graphics} % for EPS, load graphicx instead
%\usepackage{times}    % comment if you want LaTeX's default font
\usepackage{url}      % llt: nicely formatted URLs

% llt: Define a global style for URLs, rather that the default one
\makeatletter
\def\url@leostyle{%
  \@ifundefined{selectfont}{\def\UrlFont{\sf}}{\def\UrlFont{\small\bf\ttfamily}}}
\makeatother
\urlstyle{leo}


% To make various LaTeX processors do the right thing with page size.
\def\pprw{8.5in}
\def\pprh{11in}
\special{papersize=\pprw,\pprh}
\setlength{\paperwidth}{\pprw}
\setlength{\paperheight}{\pprh}
\setlength{\pdfpagewidth}{\pprw}
\setlength{\pdfpageheight}{\pprh}

% Make sure hyperref comes last of your loaded packages, 
% to give it a fighting chance of not being over-written, 
% since its job is to redefine many LaTeX commands.
\usepackage[pdftex]{hyperref}
\hypersetup{
pdftitle={SIGCHI Conference Proceedings Format},
pdfauthor={LaTeX},
pdfkeywords={SIGCHI, proceedings, archival format},
bookmarksnumbered,
pdfstartview={FitH},
colorlinks,
citecolor=black,
filecolor=black,
linkcolor=black,
urlcolor=black,
breaklinks=true,
}

% create a shortcut to typeset table headings
\newcommand\tabhead[1]{\small\textbf{#1}}


% End of preamble. Here it comes the document.
\begin{document}

\title{SIGCHI Conference Proceedings Format}

\numberofauthors{3}
\author{
  \alignauthor 1st Author Name\\
    \affaddr{Affiliation}\\
    \affaddr{Address}\\
    \email{e-mail address}\\
    \affaddr{Optional phone number}
  \alignauthor 2nd Author Name\\
    \affaddr{Affiliation}\\
    \affaddr{Address}\\
    \email{e-mail address}\\
    \affaddr{Optional phone number}    
  \alignauthor 3rd Author Name\\
    \affaddr{Affiliation}\\
    \affaddr{Address}\\
    \email{e-mail address}\\
    \affaddr{Optional phone number}
}

\maketitle

\begin{abstract}

Wearable devices are useful for recording biometric data. The ebbs, flows, mental and physical capacities of people's bodies inform their lives. When viewed over time, and organized into a useful format, this data can be seen as a type of personal narrative. It is not just a stream of unrelated data; it can be structured to have a 'narrative' arc. 

If bodies are capable of becoming sources of important narratives then to measure aspects of people's bodies, their mobility, their actions, their positioning in space and time, in a structured way, such narratives could possibly be useful for explaining people's lives to themselves.

The concept of transmedia provides a helpful structure for structuring biometric data. Transmedia enables narratives to span multiple devices and media channels. These narratives are authored by writers intentionally, with explicit and carefully engineered connection points between media platforms. Despite these transition points, the integrity of the narrative must remain intact. 

This research examines the connection between biometric data and how it might be structured into narratives similar to interactive transmedia narratives. 
\end{abstract}

\keywords{
Transmedia, wearable technology
}

\category{H.5.m.}{Information Interfaces and Presentation (e.g. HCI)}{Miscellaneous}

\section{Introduction}


Both narrative and transmedia are important, though sometimes quite hidden, aspects of modern lives. Narratives, stories that people construct to explain their actions, inform all aspects of interior and social life. Internal narratives seem important for our psychological makeup. 

Certain points in any narrative that are more important than others. For instance, in a media property called Time Tremors (a series involving pre-teen children and their cosmically evil teacher) it is when ominous, supernatural things start happening to the protagonists, such as when toy bears start talking, bugs invade a classroom or when a so-called time tremor is triggered. These are special inflection points within the narrative that demand special attention. 

In a similar fashion there are also inflection points in internal narratives; these are points in people's lives when life takes a different direction. Wearable device are useful by their ability to measure and identify these important 'inflection points.'

In transmedia the viewer has some ability to exercise choice and to customize their narrative experience. Transmedia narratives must be designed to allow for this kind of non-prescribed navigation. This gives an inherent modularity to its content.

Narratives generated by bodies as they navigate the world is similar to narratives written to span multiple media venues: both are modular, customizable, and depend on the personal experiences of the viewers. 


\section{Internal Narratives}



Internal narratives are ones were construct within ourselves. People create narratives by reflecting on the processes within their lives. This process requires that people engage conceptually and cognitively with their experience. People order and make sense of their lives in terms of narrative.

\section{Conclusion}



\section{Acknowledgments}

We would like to thank NSERC, ISTP, OCADU and others for their support.

% Balancing columns in a ref list is a bit of a pain because you
% either use a hack like flushend or balance, or manually insert
% a column break.  http://www.tex.ac.uk/cgi-bin/texfaq2html?label=balance
% multicols doesn't work because we're already in two-column mode,
% and flushend isn't awesome, so I choose balance.  See this
% for more info: http://cs.brown.edu/system/software/latex/doc/balance.pdf
%
% Note that in a perfect world balance wants to be in the first
% column of the last page.
%
% If balance doesn't work for you, you can remove that and
% hard-code a column break into the bbl file right before you
% submit:
%
% http://stackoverflow.com/questions/2149854/how-to-manually-equalize-columns-
% in-an-ieee-paper-if-using-bibtex
%
% Or, just remove \balance and give up on balancing the last page.
%
\balance

\section{References format}
References must be the same font size as other body text.
% REFERENCES FORMAT
% References must be the same font size as other body text.

\bibliographystyle{SIGCHI2015LaTex/acm-sigchi}
\bibliography{../refs.bib}
\end{document}
